\PARstart Testeamos los algoritmos en tres etapas.

\subsection{Efectos en la se\~nal y el espectro}

Graficamos el espectro original y el filtrado para se\~nales
de audio.
La idea, es mostrar que, por m\'as que ganemos
cierto rango din\'amico, no estaremos filtrando el ruido en
si, sino frecuencias que lo hacen m\'as notable.

\subsection{Buenos par\'ametros para se\~nales espec\'ificas}

Tomando un conjunto de audios e im\'agenes buscamos par\'ametros
de ambos filtros que maximicen el rango din\'amico ganado para
distintos niveles de ruido agregado.
Queremos ver cuando los filtros resultan m\'as efectivos.
Es decir, si lo son en la presencia de m\'as o menos ruido.

\subsection{Tiempo de corrida}

Para el caso de las im\'agenes comparamos el tiempo de ejecuci\'on
de los filtros est\'andar con los filtros por bloques para
distintos tama\~nos de bloque. Esperamos ver una complejidad
cuadr\'atica para bloques peque\~nos e imagenes grandes
ya que las operaciones de resoluci\'on de sistemas ser\'an
despreciables y se notar\'a m\'as que nada el recorrido de la matriz
principal, de tama\~no cuadr\'atico en funci\'on del lado.
