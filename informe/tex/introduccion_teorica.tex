\section{Introducc\'ion te\'orica}

	\PARstart Se llama factorizaci\'on SVD (\textit{singular value decomposition})
	de una matriz $A$ (de dimensiones arbitrarias) a un producto de tres matrices
	$U\Sigma T^{t} = A$. Donde $U$ y $T$ son ortogonales y $\Sigma$ tiene
	coeficientes nulos en todo elemento $\sigma_{i,j}, i\neq j$ (usualmente
	llamada diagonal, si bien sus dimensiones son iguales a las de $A$).
	Si los coeficientes de la diagonal de $\Sigma$ tienen
	la particularidad de estar ordenados de la forma $\sigma_{i,i} \geq
	\sigma_{j,j}, i<j$, entonces esta factorizaci\'on es \'unica.
	Una propiedad importante de esta factorizaci\'on es que las matrices
	$U$ y $V$ tienen como columnas los autovectores de $AA^t$ y $A^tA$
	respectivamente.
	Los elementos de la diagonal de $\Sigma$ se denominan valores singulares
	de $X$.

	La factorizaci\'on puede ser interpretada de varias formas y suele
	brindar mucha informaci\'on acerca de la matriz. En este caso queremos
	calcular las componentes principales de un cierto vector que representa
	una imagen de un d\'igito manuscrito. Esto lo logramos calculando la
	factorizaci\'on SVD de la matriz de covarianza entre p\'ixeles $X^tX$.
	Para conseguir la matriz $X$ se parte de una base de datos $x_1, \ldots, x_n$
	con im\'agenes
	de d\'igitos manuscritos de dimensiones iguales. Luego se interpreta cada
	imagen $x_i$ como un vector fila y se calcula, \'indice a \'indice,
	el vector promedio $\mu$ de todas las im\'agenes.
	La matriz $X$ se obtiene al poner como filas los vectores $x_i - \mu$
	y dividiendo por $\sqrt{n-1}$.

	Las $k$ componentes principales de una imagen dada se obtienen ralizando
	el producto interno entre el vector que representa la imagen y los
	$k$ autovectores que se corresponden con los $k$ autovalores de mayor m\'odulo
	de la matriz de covarianza entre p\'ixeles.
	Es decir, las $k$ componentes principales de una imagen $x$ son los
	primeros $k$ coeficientes del vector que resulta de multiplicar $V^tx$,
	suponiendo que se realiz\'o la factorizaci\'on SVD de $X$ de manera tal
	que los valores singulares de $X$ se encuentran ordenados de mayor a menor. 

	Una vez obtenidas las componentes principales de la imagen que quiere
	reconocerse, pueden compararse con las componentes principales de los
	d\'igitos de la base de datos de diversas maneras. T\'ipicamente se toma
	la distancia correspondiente a la norma eucl\'idea entre vectores para
	buscar la imagen cuyas componentes principales se encuentren m\'as cercanas
	a las de la imagen que quiere reconocerse.
