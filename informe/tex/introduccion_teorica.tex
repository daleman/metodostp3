\section{Introducc\'ion te\'orica}

\PARstart La factorizaci\'on SVD puede tener varias interpretaciones.
Se llama factorizaci\'on SVD de una matriz $A$ (de dimensiones arbitrarias)
a un producto de tres matrices $U\Sigma T^{t} = A$. Donde $U$ y $T$
son ortogonales y $Sigma$ tiene entradas nulas en todo elemento
$\sigma_{i,j}, i\neq j$ (usualmente llamada diagonal, si bien sus dimensiones
son iguales a las de $A$). Si las entradas de la diagonal de $\Sigma$ tienen
la particularidad de estar ordenadas de la forma $\sigma_{i,i} \geq
\sigma_{j,j}, i<j$, entonces esta factorizaci\'on es \'unica.

Esta factorizaci\'on puede ser interpretada de varias formas y suele
brindar mucha informaci\'on acerca de la matriz. En este caso queremos
calcular las componentes principales de un cierto vector que representa
una imagen de un d\'igito manuscrito.

Para esto usamos una base de datos de d\'igitos manuscritos.
Cada uno de estos d\'igitos es una imagen que la interpretamos como
un vector fila. De esta manera obtenemos una matriz de im\'agenes.
Donde la cantidad de columnas es la cantidad de pixeles en cada imagen
y la cantidad de filas es la cantidad de im\'genes.
A partir de esta base de datos nos interesa calcular la covarianza entre
cada pixel. Usamos la f\'ormula: JSDAKJSDBAD constrsdijnaskdjansjka

Las k-componentes principales de una imagen dada se obtienen ralizando
el producto interno entre el vector de la imagen y los k-autovectores
que se corresponden con los k-autovalores de mayor m\'odulo de la matriz
de covarianza.
