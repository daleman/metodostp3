\PARstart Gran parte del procesamiento de se\~nales se realiza en el
dominio transformado. Las ventajas son varias y pueden resumirse
a:

\begin{itemize}
	\item manipular el espectro de una se\~nal de forma directa
	\item realizar convoluciones en tiempo lineal en lugar
	de cuadr\'atico \footnote{Esta mejora es realmente notable
	cuando se usa una transformada r\'apida.}
\end{itemize}

En este caso estudiamos el primer punto, la manupulaci\'on de las
componentes arm\'onicas de una se\~nal discretizada. Usamos la DCT
para obtener nuestra se\~nal en la base ortogonal compuesta por
cosenos con distintas frecuencias. Estas frecuencias son divisiones
enteras de la frecuencia de muestreo\cite{paper}.

La transformada se obtiene realizando una simple multiplicaci\'on
entre la matriz cambio de base y la se\~nal original.

Para reducir el ruido de la se\~nal usamos dos filtros:
\begin{itemize}
	\item[LPF:] atenua frecuencias agudas
	\item[multiband gate:] atenua frecuencias por debajo de un cierto
	umbral
\end{itemize}

Para volver al dominio de las se\~nales aplicamos el cambio de
base inverso\cite{paper}. En este caso usamos la tecnica de plantear y
resolver el
sistema lineal dado por la matriz cambio de base y las componentes
arm\'onica de la se\~nal filtrada como se pide en \cite{enunciado}.

Utilizamos ruido normal para obtener imagenes ruidosas y poder calcular
el PSNR logrado.
