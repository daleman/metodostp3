\section{Desarrollo}

\PARstart Para obtener los autovectores de una matriz implementamos
dos estrategias:
\begin{itemize}
	\item[M\'etodo de la Potencia y deflaci\'on:] Suponiendo que
	tenemos autovalores distintos $\lambda_1 > \ldots > \lamda_n$,
	Obtenemos $\lambda_1$, el autovalor mayor, mediante
	el m\'etodo de la potencia. Luego construimos una matriz que tenga
	como autovalores $\lambda_2 > \ldots > \lambda_n \geq 0$.

	\item[M\'etodo QR para matrices sim\'etricas tridiagonales y m\'etodo
	de la potencia inversa???:] Asumiendo que partimos de una matriz
	sim\'etrica, obtenemos una matriz de hessemberg (y en este caso
	tridiagonal, por ser sim\'etrica) semejante, mediante reflexiones de
	Householder. Luego, mediante el algoritmo QR calculamos sus
	autovalores. Finalemente usamos esta aproximaci\'on de
	los autovalores para calcular los autovectores mediante el m\'etodo
	de la potencia inversa????.
\end{itemize}

A continuaci\'on se explican las implementaciones con m\'as detalle.

\subsection{M\'etodo de la potencia y deflaci\'on}
BAH

\subsection{M\'etodo QR para matrices sim\'etricas tridiagonales y m\'etodo
de la potencia inversa}

La matriz de covarianza es sim\'etrica. Utilizamos el algoritmo CITAA de CITAA
para obtener una matriz tridiagonal semejante a esta. Basicamente se obtiene
mediante transformaciones de Householder
Una vez obtenida usamos CITAA para calcular sus autovectores
