\PARstart Deseamos eliminar ruido tanto de im\'agenes como de
audios. Si bien el proceso es similar debemos realizar algunas
operaciones de manera particular en cada caso.

\subsection{Procesando audio}

Como se explica en \cite{enunciado}, a partir del vector $v$ que
representa el audio sampleado, obtenemos su tranformado $t$ realizando
la multiplicaci\'on $M \times v = t$, donde $M$ representa la
matriz cambio de base entre la can\'onica y la base
$\{1, cos(x), \ldots, cos(\frac{x}{n-1})\}$. Aqu\'i $x$ es la frecuencia
de sampleo.

Una vez obtenido el espectro $t$ se aplica un filtro, que es una funci\'on
de aridad $f : T \rightarrow T$, d\'onde $T$ representa el espacio
vectorial al cual pertenece $t$, en este caso $\mathbb{R}^n$.
De esta manera obtenemos nuestro espectro filtrado $e = f(t)$.

Finalmente realizamos la transformada inversa y obtenemos nuestro
audio filtrado $a$ resolviendo el sistema
$ M \times a = e $.
Este sistema es resuelto mediante el m\'etodo de triangulaci\'on
y sustituci\'on para atr\'as, {\em gauss con pivoteo}.

\subsection{Procesando im\'agenes}

En \cite{enunciado} se explica que la transformada $T$ de una se\~nal
$S$, una matriz cuadrada que representa los pixeles de una
im\'agen, se obtiene realizando: $ M \times S \times M^{t} = T $.
Dado que la matriz asociada a una imagen no suele ser cuadrada
recortamos la im\'agen para obtener una imagen cuadrada\footnote{
Veremos que uno de los filtros implementados no impondr\'a esta
precondici\'on en la entrada.}.

Luego transformamos y aplicamos una funci\'on similar a la
explicada anteriormente
para obtener el espectro filtrado $E$.

Finalmente debemos realizar la transformada inversa para regresar
al dominio original y obtener nuestra se\~nal filtrada $A$.
Para esto debe resolverse el sistema: $ M \times A \times M^{t} = E $.

Esto se logra notando que podemos resolverlo de a poco. Planteamos
$ Y = A \times M^{t} $ y obtenemos el sistema: $ M \times Y = E $.
Este lo resolvemos realizando $n$ veces {\em gauss}, uno por cada
columna de $E$ e $Y$.

Luego resolvemos la ecuaci\'on $ Y = A \times M^{t} $.
Para esto notamos que es equivalente a: $ Y^{t} = M \times A^{t} $.
Este sistema se resuelve de manera an\'aloga al explicado
anteriormente.

Vimos que el proceso de realizar {\em gauss} $n$ tomaba demasiado tiempo.
De hecho, ya que {\em gauss} toma un tiempo c\'ubico, estaremos hablando
de una complejidad de $O(n^4)$. Una complejidad como esta suele
no ser admisible, sobre todo si puede mejorarse.
Por este motivo realizamos la optimizaci\'on de factorizar a la matriz
cambio de base mediante la factorizaci\'on PLU.
De esta manera estaremos triangulando una \'unica vez, y la complejidad
permanecer\'a c\'ubica.

Finalmente, luego de resolver ambos sistemas, obtenemos la matriz $A$,
nuestra imagen filtrada.

\subsection{Filtrando una se\~nal}

Implementamos dos filtros que trabajan sobre el espectro
de maneras distintas.

\subsubsection{Filtro tipo Gate}

Este filtro se comporta como una compuerta multibanda. A partir de un
umbral dado en decibeles,
se reducen, por un factor dado, todas las componentes arm\'onicas que
se encuentran por debajo del umbral.

La idea es que frecuencias con una baja amplitud contribuyen m\'as
al ruido que a la se\~nal en si.

\subsubsection{Filtro tipo LPF}

Este filtro atenua por factor dado, las frecuencias agudas,
a partir de cierta frecuencia inicial.
Para esto se elige un porcentaje que
simboliza el punto del espectro a partir del cual
deben filtrarse las componentes arm\'onicas.
Esto es an\'alogo a realizar una convoluci\'on entre la se\~nal y una
{\em sinc function}, es decir, es el filtro pasa-bajos m\'as simple
que puede lograrse\cite[The Ideal Lowpass Filter]{stanford}.

La idea detr\'as de este filtro es que la frecuencias agudas tienden
a ser percibidas m\'as por el ser humano, tanto visual como auditivamente.
Por ende, si hubiese ruido, ser\'a percibido m\'as en esta parte del
espectro.

Las frecuencias agudas se asocian, visualmente, con transiciones
m\'as frecuentes en la escala
de grises. Por lo tanto reducir las frecuencias agudas resultar\'a en
una imagen con cambios m\'as suaves.

Auditivamente, la reducci\'on de frecuencias agudas suele utilizarse
para disminuir el {\em hiss} de casettes, vinilos e incluso, en etapas
de edici\'on y mezcla de audio, para limpiar guitarras electricas y voces
de componentes arm\'onicas innecesarias, que contribuyen al piso de ruido
general.

\subsubsection{Filtro por bloques}

En el caso de las imagenes notamos que, por m\'as que usemos la
factorizaci\'on PLU para reducir el tiempo de c\'omputo, el mismo
sigue siendo muy significativo.

Por esta raz\'on implementamos los filtros previamente explicados
por bloques.
Es decir, dada una imagen, tomaremos submatrices cuadradas de la
misma y las filtraremos una por una. La dimensi\'on de estas
matrices ser\'a elegida en un {\em trade-off} entre calidad del
resultado y tiempo de c\'omputo.

Notemos adem\'as, que este filtro no tendr\'a restricciones en
cuanto a las dimensiones de la imagen.

\subsection{Midiendo las mejoras}

Si bien las mejoras terminan siendo \'utiles en la medida en que sean
agradables al ser humano, es de suma utilidad poder cuantificarlas.
Para esto implementamos una funci\'on que calcula el
{\em peak signal-to-noise ratio} en funci\'on de la se\~nal original
y la filtrada.

\subsection{Ensuciando una se\~nal}

Para poder experimentar m\'as c\'omodamente realizamos una 
funci\'on para aplicar ruido a una se\~nal.
El tipo de ruido elegido es el ruido gaussiano.

Cada muestra del ruido se genera con la f\'ormula:
$$ guille tira la formula $$.
