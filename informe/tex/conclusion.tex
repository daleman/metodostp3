\section{Conclusiones}

	\PARstart Analizamos dos m\'etodos de c\'alculo de autovectores y dos (efectivos)
	de reconocimiento de d\'igitos.

	Para los autovectores el m\'etodo m\'as preciso result\'o el
	\textit{QR-potencia Inversa}. Comprobamos que los autovectores suelen
	ser muy precisos, ya que su direcci\'on no se ve practicamente alterada
	por multiplicarlos por la matriz. El m\'etodo se comporta muy bien para
	tolerancias relativamente grandes (en el tipo de matrices que analizamos)
	y no suele depender fuertemente de la tolerancia elegida. Por otro lado
	no es un m\'etodo particularmente r\'apido, y, si se admite un poco de
	error, pueden obtenerse resultados razonables bastante m\'as rapido
	con el m\'etodo \textit{potencia-deflaci\'on}. Para este caso particular
	notamos una diferencia en los resultados obtenidos al comparar los
	dos m\'etodos de obtenci\'on de autovectores. Pero las diferencias
	pueden considerarse tolerables ya son del \'orden de $\sim 0.5%$ para
	ambos m\'etodos de reconocimiento de d\'igitos.

	Con respecto al reconocimiento de d\'igitos creemos que el m\'etodo del
	promedio (al menos nuestra implementaci\'on) no es suficientemente buena
	para ninguna aplicaci\'on pr\'actica. En el mejor caso obtuvimos un
	\textit{hit rate} del $71.12%$. A lo sumo puede combinarse con un m\'etodo
	m\'as poderoso (como el de los vecinos) para hacer un \textit{double check}
	en el caso en el que el de los vecinos no llegue a un resultado concluyente
	\footnote{Por ejemplo, si los vecinos contienen cantidades similares
	de d\'igitos $a$ y d\'igitos $b$, $a\neq b$.}.
	El m\'etodo de los vecinos result\'o ser muy satisfactorio. Su mayor
	deficiencia est\'a en los d\'igitos $8$ y $9$ como mostramos en el \'ultimo
	test.
