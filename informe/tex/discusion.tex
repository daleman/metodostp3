

\subsection{Error en ciertos p\'ixeles}

Si bien implementamos los algoritmos como es explicado y los mismos
se comportan como era de esperarse, notamos que, al aplicar tanto el
filtro pasa-bajos como el de tipo compuerta algunos pixeles se saturan.

Esto se hace evidente en la esquina izquierda-superior de las im\'agenes.
Luego de investigar el problema, estamos seguros de que el fen\'omeno
proviene de el filtrado en s\'i, y no de un problema num\'erico de la
transformada.

Conjeturamos que se debe al uso de un corte muy repentino entre
se\~nales sin atenuaci\'on y se\~nales con atenuaci\'on.
Como se explica en \cite[Window Method for FIR Filter Design]{stanford},
suele transformarse la respuesta en frecuencia deseada, aplicarsele
una ventana, y luego transformarse otra vez, ahora si para modificar
el espectro de la se\~nal. Este procedimiento minimiza los artefactos
creados por el uso de una respuesta en frecuencia poco suave, como es
nuestro caso.
