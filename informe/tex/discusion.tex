\section{Discusi\'on}

	\PARstart A continuaci\'on discutimos cada uno de los resultados obtenidos en los tests
	realizados.

	\subsection{Error de los autovectores calculados}

		No nos result\'o sencillo interpretar los resultados. Esperabamos ver una
		variaci\'on m\'as significativa en las distancias. Deducimos que las instancias
		que manejamos tienen una influencia en estos resultados:
		se trata de matrices de covarianza con coeficientes realmente grandes (del orden
		de $10^4$ y m\'as). Adem\'as las matrices son considerablemente esparsas ya que
		las im\'agenes de los d\'igitos suelen contener muchos ceros, pues se trata de
		lineas sobre fondo blanco.
		Considerando lo anterior nos result\'o razonable usar una tolerancia de $0.001$
		para obtener resultados \'optimos en el caso del m\'etodo de la potencia.

		Con respecto al m\'etodo \textit{QR-potencia Inversa} llegamos a la conclusi\'on
		de que tolerancias de estos \'ordenes no tiene influencia pr\'actica en los resultados.
		Esto se debe a que el m\'etodo comienza buscando una matriz semejante que sea
		tridiagonal. Una vez obtenida esta matriz verificamos que los elementos de la diagonal
		suelen ser alg\'un orden de magnitud m\'as grandes que los de la sub-diagonal.
		Tengamos en cuenta que hasta este punto no se utiliz\'o la tolerancia ya que
		obtener la matriz de hessemberg es un m\'etodo determinista, en el sentido de que
		lleva una cantidad de pasos bien definida (que depende del tama\~no de la matriz).
		No se trata de un m\'etodo iterativo.

		Teniendo una diferencia realmente grande entre diagonal y sub-diagonal, al m\'etodo QR,
		no le toma m\'as que unas pocas iteraciones converger. En este momento la tolerancia
		entra en juego. Pero la convergencia del m\'etodo es muy r\'apida y por ende habr\'a
		muy pocas iteraciones de diferencia para dos tolerancias significativamente distintas.

		Al tener autovalores calculados con una muy buena precisi\'on, el m\'etodo de la
		potencia inversa converge casi instantaneamente\footnote{Comprobamos que no suele tomar
		m\'as de dos iteraciones. En la discusi\'on del pr\'oximo experimento se hace evidente
		que la cantidad de iteraciones que le lleva converger no depende significativamente
		de la tolerancia utilizada.}. Esto se debe a que resuelve un sistema de ecuaciones
		utilizamndo una aproximaci\'on buena de los autovalores. Al utilizar la descomposici\'on
		LU para resolver el sistema buscamos mantener el error lo m\'as acotado posible.
		Suponemos que esto es efectivo al menos en estas instancias, pues, como afirmamos
		en la secci\'on \textbf{Resultados}, el error de los autovectores es realmente
		insigificante para toda tolerancia razonable.
		
	\subsection{Diferencias en el tiempo de ejecuci\'on para obtener autovectores}

		Notamos que, como da a entender el experimento anterior, la velocidad de convergencia
		del m\'etodo \textit{QR-potencia Inversa} no depende fuertemente de la precisi\'on
		utilizada. Por el contrario el m\'etodo de la potencia parece mostrar un comportamiento
		lineal en la precisi\'on. Esto es avalado por la teor\'ia, ya que el m\'etodo de la
		potencia tiene, te\'oricamente, una convergencia lineal.

		Estos resultados cumplen con nuestras expectativas, ya que el m\'etodo \textit{QR-potencia
		Inversa} toma un tiempo de circa un orden de magnitud m\'as que el m\'etodo de la potencia.

	\subsection{Cantidad de d\'igitos reconocidos en funci\'on del m\'etodo,
	cantidad de componentes utilizada y la precisi\'on de los autovectores}
		
		Para el m\'etodo de los vecinos ponderados encontramos que la catitdad \'optima de
		vecinos es $1$. Esto muestra que el m\'etodo no es realmente efectivo, ya que,
		para un solo vecino es equivalente al m\'etodo de los vecinos est\'andar.
		Por este motivo concluimos que el metodo no es \'util.

		El m\'etodo de los vecinos est\'andar mostr\'o un mejor comportamiento al
		usar $5$ vecinos. Esto concuerda con resultados discutidos con nuestros
		compa\~nero de clase. Para el resto de los test utilizamos esta cantidad
		de vecinos.

		Para el m\'etodo de los vecinos encontramos que una buena cantidad de componentes
		principales a utilizar es $50$.
		Como deducimos del primer test no notamos una diferencia importante en el \textit{hit rate}
		cuando se var\'ia la tolerancia usando el m\'etodo de la potencia. Notamos una variaci\'on
		de $\sim 0.2\%$ al cambiar la precisi\'on. Con lo cual nos convencemos de que, utilizando
		el m\'etodo de la potencia, una tolerancia del \'orden de $0.001$ es suficiente para
		conseguir resultados \'optimos.
		Por otro lado, es importante notar que la calidad de los resultados del m\'etodo de los
		vecinos depende m\'as de la cantidad de componentes usada que de la cantidad de vecinos.

		Finalmente analizamos el m\'etodo de la distancia al promedio de las componentes
		principales. Si bien el m\'etodo es computacionalmente m\'as simple, ya que se
		compara cada d\'igito a reconocer \'unicamente con diez promedios, no muestra
		resultados satisfactorios, apenas logra superar el $70\%$ de \textit{hits}.
		Notamos que para un comportamiento \'optimo necesita del orden de la mitad de
		componentes principales que el m\'etodo de los vecinos, con lo cual el m\'etodo
		funciona mucho m\'as r\'apido.


	\subsection{Cantidad de d\'igitos reconocidos en funci\'on del m\'etodo,
	cantidad de componentes utilizada y la precisi\'on de los autovectores}

		Vemos que el tama\~no del \textit{training set} determina el porcentaje de d\'igitos bien
		reconocidos en todos los casos. Por otro lado 

		ASLDNALSDNASLKDNSKLDN


	\subsection{D\'igitos mejor reconocidos en funci\'on del m\'etodo}

		Aqu\'i se nota bien la diferencia entre los dos m\'etodos de reconocimiento de
		d\'igitos. El m\'etodo de los vecinos tiene un comportamiento casi perfecto en
		los d\'igitos $0$ y $1$, y para cualquier otro d\'igito no baja del $90\%$ de
		\textit{hits}. Por el contrario, el m\'etodo de la distancia al promedio de las
		componentes es mucho m\'as inestable. Por ejemplo vemos que con el $4$ y el $5$
		tiene un \textit{hitrate} por debajo del $50$. Este tipo de analisis puede ayudar
		a desarrollar software m\'as inteligente: sabiendo en cuales d\'igitos es efectivo
		cada m\'etodo, pueden usarse varios m\'etodos simultaneamente, y tomar una decisi\'on
		en funcion del \textit{guess} de los distintos m\'etodos.
