\documentclass[%
	%draft, %submission,
	compressed,
	%final,
	% %technote,
	%internal,
	%submitted,
	%inpress,
	%reprint,
	%
	titlepage,
	%notitlepage,
	%anonymous,
	narroweqnarray,
	inline,
	twoside,
        %invited,
	]{ieee}

\usepackage[latin1]{inputenc}			% idioma
\usepackage[spanish]{babel}
\usepackage{color}
\usepackage{colortbl}
\usepackage{amsmath}
\usepackage{amsfonts}
\usepackage{verbatim}

% -------- PARA FIGURAS
\usepackage{graphicx}
\usepackage{capt-of}
% -------- FIN FIGURAS

\usepackage{hyperref}				% urls

% -------- PARA MOSTRAR EL CODIGO DE MANERA AMENA
\usepackage{courier}
\usepackage{listings}
\lstset{ %
	language=C++,                % choose the language of the code
	basicstyle=\footnotesize\ttfamily,       % the size of the fonts that are used for the code
	numbers=left,                   % where to put the line-numbers
	numberstyle=\footnotesize,      % the size of the fonts that are used for the line-numbers
	stepnumber=1,                   % the step between two line-numbers. If it is 1 each line will be numbered
	numbersep=5pt,                  % how far the line-numbers are from the code
	backgroundcolor=\color{white},  % choose the background color. You must add \usepackage{color}
	showspaces=false,               % show spaces adding particular underscores
	showstringspaces=false,         % underline spaces within strings
	showtabs=false,                 % show tabs within strings adding particular underscores
	frame=single,           % adds a frame around the code
	tabsize=6,          % sets default tabsize to 2 spaces
	captionpos=b,           % sets the caption-position to bottom
	breaklines=true,        % sets automatic line breaking
	breakatwhitespace=false,    % sets if automatic breaks should only happen at whitespace
}
% -------- FIN CODIGO

%\hypersetup{citecolor=black}	% color citas


%--------- PARA EL ENUNCIADO DE LA CATEDRA
\parskip = 11pt

%\addtolength{\hoffset}{-1cm}
%\addtolength{\textwidth}{2cm}
 \addtolength{\voffset}{-0.5cm}
 \addtolength{\textheight}{1cm}


\newcommand{\real}{\mathbb{R}}
\newcommand{\nat}{\mathbb{N}}
\newcommand{\eme}{\mathcal{M}}
\newcommand{\emeh}{\widehat{\mathcal{M}}}
\newcommand{\ere}{\mathcal{R}}
%----------- FIN PARA EL ENUNCIADO DE LA CATEDRA


\hypersetup{			%sacar los colores horrendos de las ref
	colorlinks=false,
	pdfborder={0 0 0},
}

\begin{document}


\title[Reconocimiento de d\'igitos manuscritos]{%
	Reconocimiento de d\'igitos manuscritos a partir de la factorizaci\'on SVD\\
{\small M\'etodos Num\'ericos, Departamento de Computaci\'on, Universidad de Buenos Aires}
}

\journal{M\'etodos Num\'ericos, Departamento de Computaci\'on, Universidad de Buenos Aires}
\author[ALEMAN, G. DIEZ Y SCOCCOLA]{
Guillermo Gallardo Diez\authorinfo{gagdiez.c@gmail.com},\and{}Damian Eliel Aleman\authorinfo{damianealeman@gmail.com},\and{}y Luis Scoccola\authorinfo{luis.scoccola@gmail.com}
}


\firstpage{1}

\maketitle               
\tableofcontents

\newpage

\begin{abstract} 
	En el presente trabajo estudiaremos el reconocimiento automatizado
	de d\'igitos manuscritos mediante la comparaci\'on de las componentes
	principales de una im\'agen de un d\'igito manuscrito con las de una
	serie de imagenes de entrenamiento.
	Compararemos performance y calidad de resultados de dos m\'etodos
	distintos para hayar autovectores de matrices. i??????SLASMDL:KSD
\end{abstract}

\begin{keywords}
\end{keywords}

%----------------------------------------------------------------------
\section{Introducc\'ion te\'orica}

	\PARstart Se denomina factorizaci\'on SVD (\textit{singular value decomposition})
	de una matriz $A$ (de dimensiones arbitrarias) a la descomposici\'on de dicha matriz
	en el producto: $A = U\Sigma T^{t} $. Donde $U$ y $T$ son ortogonales 
	y $\Sigma$ tiene coeficientes nulos en todo elemento $\sigma_{i,j}, i\neq j$ (usualmente
	llamada diagonal, si bien sus dimensiones son iguales a las de $A$).
	Si los coeficientes de la diagonal de $\Sigma$ tienen
	la particularidad de estar ordenados de la forma $\sigma_{i,i} \geq
	\sigma_{j,j}, i<j$, entonces esta factorizaci\'on es \'unica.
	Una propiedad importante de esta factorizaci\'on es que las matrices
	$U$ y $V$ tienen como columnas los autovectores de $AA^t$ y $A^tA$
	respectivamente.
	Los elementos de la diagonal de $\Sigma$ se denominan valores singulares
	de $X$.

	La factorizaci\'on puede ser interpretada de varias formas y suele
	brindar mucha informaci\'on acerca de la matriz. En este caso queremos
	calcular las componentes principales de un cierto vector que representa
	una imagen de un d\'igito manuscrito. Esto lo logramos calculando la
	factorizaci\'on SVD de la matriz de covarianza entre p\'ixeles $X^tX$.
	Para conseguir la matriz $X$ se parte de una base de datos $x_1, \ldots, x_n$
	con im\'agenes
	de d\'igitos manuscritos de dimensiones iguales. Luego se interpreta cada
	imagen $x_i$ como un vector fila y se calcula, \'indice a \'indice,
	el vector promedio $\mu$ de todas las im\'agenes.
	La matriz $X$ se obtiene al poner como filas los vectores $x_i - \mu$
	y dividiendo por $\sqrt{n-1}$.

	Las $k$ componentes principales de una imagen dada se obtienen ralizando
	el producto interno entre el vector que representa la imagen y los
	$k$ autovectores que se corresponden con los $k$ autovalores de mayor m\'odulo
	de la matriz de covarianza entre p\'ixeles.
	Es decir, las $k$ componentes principales de una imagen $x$ son los
	primeros $k$ coeficientes del vector que resulta de multiplicar $V^tx$,
	suponiendo que se realiz\'o la factorizaci\'on SVD de $X$ de manera tal
	que los valores singulares de $X$ se encuentran ordenados de mayor a menor. 

	Una vez obtenidas las componentes principales de la imagen que quiere
	reconocerse, pueden compararse con las componentes principales de los
	d\'igitos de la base de datos de diversas maneras. T\'ipicamente se toma
	la distancia correspondiente a la norma eucl\'idea entre vectores para
	buscar la imagen cuyas componentes principales se encuentren m\'as cercanas
	a las de la imagen que quiere reconocerse.

 
%----------------------------------------------------------------------
\PARstart Deseamos eliminar ruido tanto de im\'agenes como de
audios. Si bien el proceso es similar debemos realizar algunas
operaciones de manera particular en cada caso.

\subsection{Procesando audio}

Como se explica en \cite{enunciado}, a partir del vector $v$ que
representa el audio sampleado, obtenemos su tranformado $t$ realizando
la multiplicaci\'on $M \times v = t$, donde $M$ representa la
matriz cambio de base entre la can\'onica y la base
$\{1, cos(x), \ldots, cos(\frac{x}{n-1})\}$. Aqu\'i $x$ es la frecuencia
de sampleo.

Una vez obtenido el espectro $t$ se aplica un filtro, que es una funci\'on
de aridad $f : T \rightarrow T$, d\'onde $T$ representa el espacio
vectorial al cual pertenece $t$, en este caso $\mathbb{R}^n$.
De esta manera obtenemos nuestro espectro filtrado $e = f(t)$.

Finalmente realizamos la transformada inversa y obtenemos nuestro
audio filtrado $a$ resolviendo el sistema
$ M \times a = e $.
Este sistema es resuelto mediante el m\'etodo de triangulaci\'on
y sustituci\'on para atr\'as, {\em gauss con pivoteo}.

\subsection{Procesando im\'agenes}

En \cite{enunciado} se explica que la transformada $T$ de una se\~nal
$S$, una matriz cuadrada que representa los pixeles de una
im\'agen, se obtiene realizando: $ M \times S \times M^{t} = T $.
Dado que la matriz asociada a una imagen no suele ser cuadrada
recortamos la im\'agen para obtener una imagen cuadrada\footnote{
Veremos que uno de los filtros implementados no impondr\'a esta
precondici\'on en la entrada.}.

Luego transformamos y aplicamos una funci\'on similar a la
explicada anteriormente
para obtener el espectro filtrado $E$.

Finalmente debemos realizar la transformada inversa para regresar
al dominio original y obtener nuestra se\~nal filtrada $A$.
Para esto debe resolverse el sistema: $ M \times A \times M^{t} = E $.

Esto se logra notando que podemos resolverlo de a poco. Planteamos
$ Y = A \times M^{t} $ y obtenemos el sistema: $ M \times Y = E $.
Este lo resolvemos realizando $n$ veces {\em gauss}, uno por cada
columna de $E$ e $Y$.

Luego resolvemos la ecuaci\'on $ Y = A \times M^{t} $.
Para esto notamos que es equivalente a: $ Y^{t} = M \times A^{t} $.
Este sistema se resuelve de manera an\'aloga al explicado
anteriormente.

Vimos que el proceso de realizar {\em gauss} $n$ tomaba demasiado tiempo.
De hecho, ya que {\em gauss} toma un tiempo c\'ubico, estaremos hablando
de una complejidad de $O(n^4)$. Una complejidad como esta suele
no ser admisible, sobre todo si puede mejorarse.
Por este motivo realizamos la optimizaci\'on de factorizar a la matriz
cambio de base mediante la factorizaci\'on PLU.
De esta manera estaremos triangulando una \'unica vez, y la complejidad
permanecer\'a c\'ubica.

Finalmente, luego de resolver ambos sistemas, obtenemos la matriz $A$,
nuestra imagen filtrada.

\subsection{Filtrando una se\~nal}

Implementamos dos filtros que trabajan sobre el espectro
de maneras distintas.

\subsubsection{Filtro tipo Gate}

Este filtro se comporta como una compuerta multibanda. A partir de un
umbral dado en decibeles,
se reducen, por un factor dado, todas las componentes arm\'onicas que
se encuentran por debajo del umbral.

La idea es que frecuencias con una baja amplitud contribuyen m\'as
al ruido que a la se\~nal en si.

\subsubsection{Filtro tipo LPF}

Este filtro atenua por factor dado, las frecuencias agudas,
a partir de cierta frecuencia inicial.
Para esto se elige un porcentaje que
simboliza el punto del espectro a partir del cual
deben filtrarse las componentes arm\'onicas.
Esto es an\'alogo a realizar una convoluci\'on entre la se\~nal y una
{\em sinc function}, es decir, es el filtro pasa-bajos m\'as simple
que puede lograrse\cite[The Ideal Lowpass Filter]{stanford}.

La idea detr\'as de este filtro es que la frecuencias agudas tienden
a ser percibidas m\'as por el ser humano, tanto visual como auditivamente.
Por ende, si hubiese ruido, ser\'a percibido m\'as en esta parte del
espectro.

Las frecuencias agudas se asocian, visualmente, con transiciones
m\'as frecuentes en la escala
de grises. Por lo tanto reducir las frecuencias agudas resultar\'a en
una imagen con cambios m\'as suaves.

Auditivamente, la reducci\'on de frecuencias agudas suele utilizarse
para disminuir el {\em hiss} de casettes, vinilos e incluso, en etapas
de edici\'on y mezcla de audio, para limpiar guitarras electricas y voces
de componentes arm\'onicas innecesarias, que contribuyen al piso de ruido
general.

\subsubsection{Filtro por bloques}

En el caso de las imagenes notamos que, por m\'as que usemos la
factorizaci\'on PLU para reducir el tiempo de c\'omputo, el mismo
sigue siendo muy significativo.

Por esta raz\'on implementamos los filtros previamente explicados
por bloques.
Es decir, dada una imagen, tomaremos submatrices cuadradas de la
misma y las filtraremos una por una. La dimensi\'on de estas
matrices ser\'a elegida en un {\em trade-off} entre calidad del
resultado y tiempo de c\'omputo.

Notemos adem\'as, que este filtro no tendr\'a restricciones en
cuanto a las dimensiones de la imagen.

\subsection{Midiendo las mejoras}

Si bien las mejoras terminan siendo \'utiles en la medida en que sean
agradables al ser humano, es de suma utilidad poder cuantificarlas.
Para esto implementamos una funci\'on que calcula el
{\em peak signal-to-noise ratio} en funci\'on de la se\~nal original
y la filtrada.

\subsection{Ensuciando una se\~nal}

Para poder experimentar m\'as c\'omodamente realizamos una 
funci\'on para aplicar ruido a una se\~nal.
El tipo de ruido elegido es el ruido gaussiano.

Cada muestra del ruido se genera con la f\'ormula:
$$ guille tira la formula $$.


%----------------------------------------------------------------------
\section{Resultados}

\subsection{Efectos en la se\~nal y el espectro}

\subsection{Buenos par\'ametros para se\~nales espec\'ificas}

\subsubsection{Sonidos}

!!!!!!!!!!!!!!!!!!!!!!!!!!1

\subsubsection{Im\'agenes}

Aqu\'i presentamos un gr\'afico que muestra la relaci\'on entre
ruido introducido y mejora del PSNR.
Es importante notar que cada uno de estos resultados fue el trabajo
de una optimizaci\'on manual\footnote{Ver el ap\'endice de tablas
para los resultados num\'ericos y los par\'ametros utilizados},
y que si bien pensamos que puede
ser automatizada, no nos result\'o evidente la manera de hacerlo.
Dado que deben optimizarse dos par\'ametros simultaneamente y
las funciones a optimizar parecen registrar varios m\'inimos
locales.

Las im\'genes usadas fueron \texttt{Red\_cuadrada.pgm} y
\texttt{masterVoice\_cuadrada.pgm}.
Dado que la primera es de un tama\~no peque\~no utilizamos
unicamente los filtros est\'andar. Para la segunda utilizamos
tambi\'en los algoritmos por bloques.

\includegraphics[width=14cm]{red_psnr.png}

{\center \textsc{Decibeles ganados - porcentaje de ruido introducido}\\
	Im\'agen \texttt{Red\_cuadrada.png} }


\includegraphics[width=14cm]{voice_psnr.png}
{ \center \textsc{Decibeles ganados - porcentaje de ruido introducido}\\
  \center Im\'agen \texttt{masterVoice\_cuadrada.png} }

\subsection{Tiempo de corrida}

La diferencia, en tiempo, entre trabajar con el espectro de toda la
im\'agen y filtrar de a bloques se ve en esta im\'agen:

AKSJDAKJSDNASKJN

Aqu\'i se comparan los tiempos de ejecuci\'on de un filtro por bloques
para distintos tama\~nos de bloque\footnote{En ambos casos no se hace
distinci\'on entre los filtros de umbral o
pasa-bajos, ya que esto no es relevante para la toma de tiempos.}:

JANSDKAJNDKAJNDKASJDN




%----------------------------------------------------------------------
\section{Discusi\'on}

\section{Discusi\'on}

	\PARstart A continuaci\'on discutimos cada uno de los resultados obtenidos en los tests
	realizados.

	\subsection{Error de los autovectores calculados}

		No nos result\'o sencillo interpretar los resultados. Esperabamos ver una
		variaci\'on m\'as significativa en las distancias. Deducimos que las instancias
		que manejamos tienen una influencia en estos resultados:
		se trata de matrices de covarianza con coeficientes realmente grandes (del orden
		de $10^4$ y m\'as). Adem\'as las matrices son considerablemente esparsas ya que
		las im\'agenes de los d\'igitos suelen contener muchos ceros, pues se trata de
		lineas sobre fondo blanco.
		Considerando lo anterior nos result\'o razonable usar una tolerancia de $0.001$
		para obtener resultados \'optimos en el caso del m\'etodo de la potencia.

		Con respecto al m\'etodo \textit{QR-potencia Inversa} llegamos a la conclusi\'on
		de que tolerancias de estos \'ordenes no tiene influencia pr\'actica en los resultados.
		Esto se debe a que el m\'etodo comienza buscando una matriz semejante que sea
		tridiagonal. Una vez obtenida esta matriz verificamos que los elementos de la diagonal
		suelen ser alg\'un orden de magnitud m\'as grandes que los de la sub-diagonal.
		Tengamos en cuenta que hasta este punto no se utiliz\'o la tolerancia ya que
		obtener la matriz de hessemberg es un m\'etodo determinista, en el sentido de que
		lleva una cantidad de pasos bien definida (que depende del tama\~no de la matriz).
		No se trata de un m\'etodo iterativo.

		Teniendo una diferencia realmente grande entre diagonal y sub-diagonal, al m\'etodo QR,
		no le toma m\'as que unas pocas iteraciones converger. En este momento la tolerancia
		entra en juego. Pero la convergencia del m\'etodo es muy r\'apida y por ende habr\'a
		muy pocas iteraciones de diferencia para dos tolerancias significativamente distintas.

		Al tener autovalores calculados con una muy buena precisi\'on, el m\'etodo de la
		potencia inversa converge casi instantaneamente\footnote{Comprobamos que no suele tomar
		m\'as de dos iteraciones. En la discusi\'on del pr\'oximo experimento se hace evidente
		que la cantidad de iteraciones que le lleva converger no depende significativamente
		de la tolerancia utilizada.}. Esto se debe a que resuelve un sistema de ecuaciones
		utilizamndo una aproximaci\'on buena de los autovalores. Al utilizar la descomposici\'on
		LU para resolver el sistema buscamos mantener el error lo m\'as acotado posible.
		Suponemos que esto es efectivo al menos en estas instancias, pues, como afirmamos
		en la secci\'on \textbf{Resultados}, el error de los autovectores es realmente
		insigificante para toda tolerancia razonable.
		
	\subsection{Diferencias en el tiempo de ejecuci\'on para obtener autovectores}

		Notamos que, como da a entender el experimento anterior, la velocidad de convergencia
		del m\'etodo \textit{QR-potencia Inversa} no depende fuertemente de la precisi\'on
		utilizada. Por el contrario el m\'etodo de la potencia parece mostrar un comportamiento
		lineal en la precisi\'on. Esto es avalado por la teor\'ia, ya que el m\'etodo de la
		potencia tiene, te\'oricamente, una convergencia lineal.

		Estos resultados cumplen con nuestras expectativas, ya que el m\'etodo \textit{QR-potencia
		Inversa} toma un tiempo de circa un orden de magnitud m\'as que el m\'etodo de la potencia.

	\subsection{Cantidad de d\'igitos reconocidos en funci\'on del m\'etodo,
	cantidad de componentes utilizada y la precisi\'on de los autovectores}
		
		Para el m\'etodo de los vecinos ponderados encontramos que la catitdad \'optima de
		vecinos es $1$. Esto muestra que el m\'etodo no es realmente efectivo, ya que,
		para un solo vecino es equivalente al m\'etodo de los vecinos est\'andar.
		Por este motivo concluimos que el metodo no es \'util.

		El m\'etodo de los vecinos est\'andar mostr\'o un mejor comportamiento al
		usar $5$ vecinos. Esto concuerda con resultados discutidos con nuestros
		compa\~nero de clase. Para el resto de los test utilizamos esta cantidad
		de vecinos.

		Para el m\'etodo de los vecinos encontramos que una buena cantidad de componentes
		principales a utilizar es $50$.
		Como deducimos del primer test no notamos una diferencia importante en el \textit{hit rate}
		cuando se var\'ia la tolerancia usando el m\'etodo de la potencia. Notamos una variaci\'on
		de $\sim 0.2\%$ al cambiar la precisi\'on. Con lo cual nos convencemos de que, utilizando
		el m\'etodo de la potencia, una tolerancia del \'orden de $0.001$ es suficiente para
		conseguir resultados \'optimos.
		Por otro lado, es importante notar que la calidad de los resultados del m\'etodo de los
		vecinos depende m\'as de la cantidad de componentes usada que de la cantidad de vecinos.

		Finalmente analizamos el m\'etodo de la distancia al promedio de las componentes
		principales. Si bien el m\'etodo es computacionalmente m\'as simple, ya que se
		compara cada d\'igito a reconocer \'unicamente con diez promedios, no muestra
		resultados satisfactorios, apenas logra superar el $70\%$ de \textit{hits}.
		Notamos que para un comportamiento \'optimo necesita del orden de la mitad de
		componentes principales que el m\'etodo de los vecinos, con lo cual el m\'etodo
		funciona mucho m\'as r\'apido.


	\subsection{Cantidad de d\'igitos reconocidos en funci\'on del m\'etodo,
	cantidad de componentes utilizada y la precisi\'on de los autovectores}

		Vemos que el tama\~no del \textit{training set} determina el porcentaje de d\'igitos bien
		reconocidos en todos los casos. Por otro lado 

		ASLDNALSDNASLKDNSKLDN


	\subsection{D\'igitos mejor reconocidos en funci\'on del m\'etodo}

		Aqu\'i se nota bien la diferencia entre los dos m\'etodos de reconocimiento de
		d\'igitos. El m\'etodo de los vecinos tiene un comportamiento casi perfecto en
		los d\'igitos $0$ y $1$, y para cualquier otro d\'igito no baja del $90\%$ de
		\textit{hits}. Por el contrario, el m\'etodo de la distancia al promedio de las
		componentes es mucho m\'as inestable. Por ejemplo vemos que con el $4$ y el $5$
		tiene un \textit{hitrate} por debajo del $50$. Este tipo de analisis puede ayudar
		a desarrollar software m\'as inteligente: sabiendo en cuales d\'igitos es efectivo
		cada m\'etodo, pueden usarse varios m\'etodos simultaneamente, y tomar una decisi\'on
		en funcion del \textit{guess} de los distintos m\'etodos.


%----------------------------------------------------------------------
\section{Conclusiones}

\section{Conclusiones}

	\PARstart Analizamos dos m\'etodos de c\'alculo de autovectores y dos (efectivos)
	de reconocimiento de d\'igitos.

	Para los autovectores el m\'etodo m\'as preciso result\'o el
	\textit{QR-potencia Inversa}. Comprobamos que los autovectores suelen
	ser muy precisos, ya que su direcci\'on no se ve practicamente alterada
	por multiplicarlos por la matriz. El m\'etodo se comporta muy bien para
	tolerancias relativamente grandes (en el tipo de matrices que analizamos)
	y no suele depender fuertemente de la tolerancia elegida. Por otro lado
	no es un m\'etodo particularmente r\'apido, y, si se admite un poco de
	error, pueden obtenerse resultados razonables bastante m\'as rapido
	con el m\'etodo \textit{potencia-deflaci\'on}. Para este caso particular
	notamos una diferencia en los resultados obtenidos al comparar los
	dos m\'etodos de obtenci\'on de autovectores. Pero las diferencias
	pueden considerarse tolerables ya son del \'orden de $\sim 0.5%$ para
	ambos m\'etodos de reconocimiento de d\'igitos.

	Con respecto al reconocimiento de d\'igitos creemos que el m\'etodo del
	promedio (al menos nuestra implementaci\'on) no es suficientemente buena
	para ninguna aplicaci\'on pr\'actica. En el mejor caso obtuvimos un
	\textit{hit rate} del $71.12%$. A lo sumo puede combinarse con un m\'etodo
	m\'as poderoso (como el de los vecinos) para hacer un \textit{double check}
	en el caso en el que el de los vecinos no llegue a un resultado concluyente
	\footnote{Por ejemplo, si los vecinos contienen cantidades similares
	de d\'igitos $a$ y d\'igitos $b$, $a\neq b$.}.
	El m\'etodo de los vecinos result\'o ser muy satisfactorio. Su mayor
	deficiencia est\'a en los d\'igitos $8$ y $9$ como mostramos en el \'ultimo
	test.


%----------------------------------------------------------------------

\begin{thebibliography}{1}

\bibitem{enunciado}
	C\'atedra de M\'etodos num\'ericos,\\
	\newblock {\em Tercer Trabajo Pr\'actico},\\
	\newblock Primer cuatrimestre 2013

\bibitem{burden}
	Richard L. Burden, \\
	\newblock {\em Numerical Analysis}, \\
	\newblock 9th ed.

\end{thebibliography}

%----------------------------------------------------------------------

\newpage

\section{Ap\'endices}

\subsection{Enunciado}
{\bf Introducci\'on}

La Transformada Discreta del Coseno  (DCT, por sus siglas en ingl\'es) es una herramienta que nos permite representar cualquier se\~nal en el plano de las frecuencias. Dado que es utilizada por el est\'andar de compresi\'on de im\'agenes JPEG y formato de video MPEG, se encuentra implementada en m\'as lugares de lo que pensamos: en cada c\'amara digital o tel\'efono m\'ovil. 
La DCT no solo tiene aplicaciones al mundo de la compresi\'on (donde los valores transformados pueden ser codificados de forma eficiente), sino tambi\'en al procesamiento: el an\'alisis de qu\'e frecuencias est\'an presentes en las se\~nales es esencial en ciertos contextos de aplicaci\'on.

La idea intuitiva de esta transformada, en el plano continuo, consiste en representar una funci\'on $f: \mathbb{R} \rightarrow \mathbb{R}$ en la base de funciones $\mathcal{B}=\{1, \cos(x), \cos(2x),...\}$.
En el plano discreto, la DCT se corresponde a un cambio de base: cada una de las funciones de la base $\mathcal{B}$ se discretiza en ciertos puntos pasando a ser una base de vectores en $\mathbb{R}^n$, (donde $n$ es la dimensi\'on del vector o se\~nal a transformar).
Es decir, dado un vector o se\~nal $x\in\mathbb{R}^n$ existe una matriz $M\in\mathbb{R}^{n\times n}$ de cambio de base que define la transformada DCT, donde $y=Mx$ es el vector o se\~nal transformado al espacio de frecuencias por la DCT (ver ap\'endice). Esta operaci\'on es f\'acilmente extensible a se\~nales de dos dimensiones (ver ap\'endice).


{\bf Enunciado}

El objetivo del trabajo es eliminar ruido sobre una se\~nal ruidosa $x\in\real^{n}$. Para ello se realiza el siguiente proceso: 
\begin{enumerate}
\item  $y:=Mx$ [Transformar usando Ec.~() de Ap.]
\item $\tilde{y} := f(y)$ [Modificar]
\item Resolver $M \tilde{x} = \tilde{y}$ [Reconstruir]
\end{enumerate}

 
Una forma de medir la calidad visual de la se\~nal reconstruida $\tilde{x}$, es a trav\'es del PSNR ({\em Peak Signal-to-Noise Ratio}).
EL PSNR es una m\'etrica `perceptual' (acorde a lo que perciben los humanos) y nos da una forma de medir la calidad de una imagen perturbada, siempre y cuando se cuente con la se\~nal original. 
Cuanto mayor es el PSNR, mayor es la calidad de la imagen. La unidad de medida es el decibel (db) y se considera que una diferencia de 0.5 db ya es notada por la vista humana. El PSNR se define como:
$$
\mathit{PSNR} = 10 \cdot \log_{10} \left( \frac{\mathit{MAX}^2_x}{\mathit{ECM}} \right)
$$
donde $\mathit{MAX}_x$ define el rango m\'aximo de la se\~nal (en caso de entradas de 8 bits sin signo, ser\'ia 255) y \emph{ECM} es el {\em error cuadr\'atico medio}, definido como:
$ \frac{1}{n} \sum_{i}{(x_{i} - \tilde{x}_{i})^2} $,
donde $n$ es la cantidad de elementos de la se\~nal, $x$ es la se\~nal original y $\tilde{x}$ es la se\~nal recuperada.

En la implementaci\'on realizada deben llevar a cabo los siguientes experimentos:
\begin{itemize}
\item Para varias se\~nales con distintos niveles de ruido, se deber\'an experimentar con al menos 2 estrategias (definiendo $f$ de forma conveniente) para modificar la se\~nal transformada $y$ (paso 2) con el objetivo de que la se\~nal recuperada $\tilde{x}$ contenga menos ruido; se deber\'an extraer conclusiones en cuanto a la calidad de la se\~nal recuperada, en funci\'on de la estrategia utilizada.
\item Se deber\'an repetir los anteriores experimentos tambi\'en sobre im\'agenes adaptando el m\'etodo para aplicar la transformada DCT en dos dimensiones seg\'un se explica en la ap\'endice. 
\item {\bf (Opcional)} Se deber\'a analizar la aplicaci\'on de la DCT `por bloques' sobre im\'agenes. Por ejemplo, si tenemos una imagen de $64\times 64$ p\'ixeles podemos subdividirla en: 4 bloques de $32\times 32$, o 16 bloques de $16\times 16$, o 64 bloques de $8\times 8$, y aplicar la DCT en 2D sobre cada uno de los bloques (considerar un tama\~no m\'inimo de $8\times 8$ para cada bloque).\\ 
Elegir una estrategia utilizada para se\~nales unidimiensionales y sacar conclusiones respondiendo a los siguientes interrogantes (realizando experimentos que justifiquen la respuesta): ?`Es lo mismo eliminar ruido sobre la imagen entera que de a bloques? ?`Qu\'e forma es m\'as conveniente en cuanto a la calidad visual? ?`Qu\'e forma es m\'as r\'apida? 
\end{itemize}

{\bf Formatos de archivos de entrada}

Las se\~nales ser\'an le\'idas de un archivo de texto en cuya primer l\'inea figuran la cantidad de datos y en la l\'inea siguiente se encuentran los datos en ASCII separados por espacios. Para leer y escribir im\'agenes sugerimos utilizar el formato {\em raw} binario \texttt{.pgm}\footnote{\url{http://netpbm.sourceforge.net/doc/pgm.html}}. 
El mismo es muy sencillo de implementar y compatible con muchos gestores de fotos\footnote{XnView \url{http://www.xnview.com/}} y Matlab.

\vskip 0.5 cm
\hrule
\vskip 0.1 cm

{\bf Fecha de entrega:} 
\begin{itemize}
\item \textsl{Formato electr\'onico:} jueves 16 de mayo de 2013, hasta las 23:59 hs., enviando el trabajo (informe+c\'odigo) a \texttt{metnum.lab@gmail.com}. El subject del email debe comenzar con el texto \verb|[TP2]| seguido de la lista de apellidos de los integrantes del grupo. 
\item \textsl{Formato f\'isico:} viernes 17 de mayo de 2013, de 18 a 20hs (en la clase del labo).
\end{itemize}


{\bf Transformada Coseno Discreta} 

Para generar la matriz $M\in\real^{n\times n}$ que define la transformada de Coseno Discreta definimos: 
\begin{itemize}
 \item Vector de frecuencias:  $g = \left(\begin{array}{c} 0 \\ 1 \\ \vdots \\ n-1 \end{array} \right)$
 \item Vector de muestreo: $ s= {\displaystyle \frac{\pi}{n} }\left(\begin{array}{c} \frac{1}{2} \\1+\frac{1}{2} \\  \vdots \\ (n-1)+\frac{1}{2} \end{array} \right) $
 \item Constante de normalizaci\'on: $C(k) = \left\{ \begin{array}{lr}\sqrt{\frac{1}{n}} & k=1 \\ \sqrt{\frac{2}{n}} & k > 1 \end{array} \right.$
\end{itemize}

Siendo $T=\cos(g\cdot s^t)$ la matriz resultante de aplicarle el coseno a cada elemento de la matriz $g\cdot s^t$,
finalmente definimos, $\widehat{M}_{i,j} = C(i) \cdot T_{i,j}$

Para obtener una versi\'on entera de la transformaci\'on que define la matriz $M$, la cual ser\'a aplicada a se\~nales (o vectores) enteras en el rango $[0,q]$, definimos:
\begin{equation}
M = \left\lfloor \frac{q \widehat{M} + 1}{2}  \right\rfloor \label{eq:dctint}
\end{equation} 
donde $\left\lfloor\cdot\right\rfloor$ indica la parte entera inferior\footnote{El redondeo de un n\'umero $m$ pude definirse como $\lfloor m + 1/2 \rfloor$. Luego, $M$ se define como el redondeo de $\widehat{M}\cdot(q/2)$.}. (Es decir, escalamos los elementos de la matriz $M$ por $q/2$ y luego redondeamos los valores.)

{\bf Extensi\'on a 2D}
Dada una matriz $B\in\real^{n\times n}$, podemos extender f\'acilmente la transformada DCT a se\~nales de dos dimensiones. Para ello, aplicamos la transformaci\'on por filas y por columnas: $ M\, B\, M^t $



\newpage

\subsection{C\'odigo relevante y modo de uso}

	\subsubsection{Conversi\'on de datos usando Matlab}
		Las instancias utilizadas fueron bajadas de \url{http://yann.lecun.com/exdb/mnist/}.
		Debe utilizarse el \textit{script} \texttt{generarPuntoDat.sh} que se
		encuentra en \texttt{imagenes/MNIST_Matlab}
		para convertir las instancias bajadas a archivos legibles por
		nuestra implementaci\'on. Al ejecutar el \textit{script} se
		ver\'a el modo de uso.

	\subsubsection{Modo de uso del programa}
		Si bien al ejecutar el programa \texttt{tp3.exe} que se encuentra
		en \texttt{ejecutables/} se imprimen las instrucciones de uso
		cabe destacar algunos puntos:

	\subsubsection{Clase \textit{Reconocedor}}
		Esta clase permite utilizar cualquiera de los m\'etodos para el calculo
		de autovalores y combinarlo con cualquiera de los m\'etodo de reconocimiento
		de d\'igitos.
		
		\lstinputlisting{../../src_cpp/reconocedor.cpp}

	\subsubsection{Clase \textit{Matriz}}

		\lstinputlisting{../../src_cpp/matriz.h}


\newpage

\subsection{Tablas}
\subsubsection{Tabla A.}
    
    Promedio tiempo corrida m\'etodos:

    \vspace{5mm}
    \centerline{\includegraphics[width=7cm]{img/tiempoPromedio.png}}
    \vspace{5mm}

\subsubsection{Tabla B.}
    
    Porcentaje de digitos reconocidos Vecinos Ponderados:

    \vspace{5mm}
    \centerline{\includegraphics[width=14cm]{img/kVecinosPonderadosPs.png}}

\subsubsection{Tabla C.}
    
    Porcentaje de digitos reconocidos Vecinos:

    \vspace{5mm}
    \centerline{\includegraphics[width=14cm]{img/kVecinosPs.png}}

\subsubsection{Tabla D.}
    
    Cantidad de componentes con M\'etodo QR Potencia Inversa Kvecinos

    \vspace{5mm}
    \centerline{\includegraphics[width=5cm]{img/QrKvecinos.png}}


\subsubsection{Tabla E.}
    
    Cantidad de componentes con M\'etodo QR Potencia Simple Kvecinos

    \vspace{5mm}
    \centerline{\includegraphics[width=5cm]{img/QRDigitosMEdios.png}}

\subsubsection{Tabla F.}
	
	Hitrate M\'etodos

	\vspace{5mm}
	\centerline{\includegraphics[width=7cm]{img/digitosBam_tabla.png}}



%----------------------------------------------------------------------

\end{document}
